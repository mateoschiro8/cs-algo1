\documentclass{article}
\input{Algo1Macros}
\usepackage{caratula} 

\begin{document}

%Caratula
\titulo{Laboratorio 01 - Latex}
\subtitulo{Algo 1}
\fecha{03/09/2022}
\materia{Algoritmos y Estructuras de datos 1}
\integrante{Schiro, Mateo}{657/22}{mateo.schiro8@gmail.com}

\maketitle
%Creación de indice
\tableofcontents
\newpage

\section*{Ejercicios Latex - Laboratorio Algo 1}

\subsection*{Ejercicio 1}

El factorial de un entero positivo n se define como: $ n! = \prod_{i = 1}^n i $  \par
\medskip El factorial de 5 es: $ 5! = \prod_{i = 1}^5 i = 1 \times 2 \times 3 \times 4 \times 5 = 120 $

\section*{Especificación}

\subsection*{Ejercicio 2}

\begin{proc}{factorial}{\In n: \ent, \Out result: \ent}{}
\pre{n \geq 0}
\post{(n = 0 \implica result = 1) $ $ \wedge $ $ (n > 0 \implica result = \prod_{k = 1}^n k)}
\end{proc}

\subsection*{Ejercicio 3}

\pred{todosPrimos}{s: \TLista \ent}{\\
(\forall i:\ent)(0 \leq i < \longitud s \implicaLuego esPrimo(s[i]))\\
}

\medskip \pred{alMenosUnPrimo}{s: \TLista \ent}{
(\exists i:\ent)(0 \leq i < \longitud s \yLuego esPrimo(s[i]))}

\subsection*{Ejercicio 4}

aux sumaPrimos (s: \TLista \ent) : \ent = $\sum_{i = 0}^ \longitud s $ if $esPrimo(s[i])$ then $s[i]$ else 0 fi;

\end{document}
